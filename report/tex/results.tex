% ______________________________________________________________________________
%
% DVG001 -- Introduktion till Linux och små nätverk
%                              Inlämningsuppgift #5
% ~~~~~~~~~~~~~~~~~~~~~~~~~~~~~~~~~~~~~~~~~~~~~~~~~
% Author:   Jonas Sjöberg
%           tel12jsg@student.hig.se
%
% Date:     2016-04-27 -- 2016-05-09
%
% License:  Creative Commons Attribution 4.0 International (CC BY 4.0)
%           <http://creativecommons.org/licenses/by/4.0/legalcode>
%           See LICENSE.md for additional licensing information.
% ______________________________________________________________________________


\section{Resultat}
Resultatet av laborationen är att vår laborationsmiljö fått ett grundläggande
skydd och dessutom möjlighet att agera både webb- och filserver.

% ~~~~~~~~~~~~~~~~~~~~~~~~~~~~~~~~~~~~~~~~~~~~~~~~~~~~~~~~~~~~~~~~~~~~~~~~~~~~~~
\section{Diskussion}
Som under tidigare laborationer kan det vara väldigt mycket detaljer och
fallgropar som kräver vana och upprepad användning med verktyg och protokoll
för att verkligen bygga upp en god kunskap om hur systemen verkligen fungerar.

% ~~~~~~~~~~~~~~~~~~~~~~~~~~~~~~~~~~~~~~~~~~~~~~~~~~~~~~~~~~~~~~~~~~~~~~~~~~~~~~
\section{Slutsatser}
Generellt tycker jag att resultaten varit goda, trots att jag själv privat
oftast t.ex. väljer att använda \texttt{gufw}, den grafiska frontenden till
\texttt{ufw} för enkelhetens skull, och därför har väldigt lite vana vid att
konfigurera just brandväggar på det vis som gjorts under laborationen. Dessutom
är det något jag oftast gör väldigt sällan, oftast en gång per installation av
ett operativsystem, så några tillfällen att verkligen bekanta sig med verktygen
och processen brukar inte presentera sig, till skillnad då mot någon som jobbar
med att administrera system dagligen. Även om jag anser mig vara något av en
''power-user'' med generellt goda kunskaper så har jag relativt lite erfarenhet
av nätverksadministration.
