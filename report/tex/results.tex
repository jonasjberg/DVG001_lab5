% ______________________________________________________________________________
%
% DVG001 -- Introduktion till Linux och små nätverk
%                              Inlämningsuppgift #5
% ~~~~~~~~~~~~~~~~~~~~~~~~~~~~~~~~~~~~~~~~~~~~~~~~~
% Author:   Jonas Sjöberg
%           tel12jsg@student.hig.se
%
% Date:     2016-04-27 -- 2016-05-09
%
% License:  Creative Commons Attribution 4.0 International (CC BY 4.0)
%           <http://creativecommons.org/licenses/by/4.0/legalcode>
%           See LICENSE.md for additional licensing information.
% ______________________________________________________________________________


\section{Resultat}
Resultatet av laborationen är att vår laborationsmiljö fått ett grundläggande
skydd och dessutom möjlighet att agera både webb- och filserver.

% ~~~~~~~~~~~~~~~~~~~~~~~~~~~~~~~~~~~~~~~~~~~~~~~~~~~~~~~~~~~~~~~~~~~~~~~~~~~~~~
\section{Diskussion}
Som under tidigare laborationer kan många problem, detaljer och fallgropar
kräva vana och upprepad användning av verktyg och protokoll, för en verkligt
god kunskap om hur systemen fungerar. Mycket av det som krävs kan verka
godtyckligt och onödigt pillrigt, och det känns många gånger tydligt att
systemen byggts upp gradvis under väldigt många år. Samtidigt känner jag mig
mycket hemma i \texttt{UNIX}-miljön och skulle inte vilja byta ut det mot något
annat. Jag märker också att mitt synsätt blir helt annorlunda över tid, i 
och med att jag övar på att använda vissa program eller system. De ibland
svårtillgängliga, esoteriska, kryptiska kommandon och program en stöter på i
\texttt{UNIX}-miljön, blir med tid och övning fullständigt självklara och många
gånger kan den långa bakomliggande historien bidra till en ännu djupare
fascination för hela \texttt{UNIX}-kulturen.

Jag tycker att laborationen ger en bra introduktion och riktlinjer för fortsatt
fördjupning, som antagligen är nödvändig för att med tid verkligen greppa en
helhet.


% ~~~~~~~~~~~~~~~~~~~~~~~~~~~~~~~~~~~~~~~~~~~~~~~~~~~~~~~~~~~~~~~~~~~~~~~~~~~~~~
\section{Slutsatser}
Generellt tycker jag att resultaten varit goda, trots att jag själv privat
oftast t.ex. väljer att använda \texttt{gufw}, den grafiska frontenden till
\texttt{ufw} för enkelhetens skull, och därför har väldigt lite vana vid att
konfigurera just brandväggar på det vis som gjorts under laborationen. Dessutom
är det något jag oftast gör väldigt sällan, oftast en gång per installation av
ett operativsystem, så några tillfällen att verkligen bekanta sig med verktygen
och processen brukar inte presentera sig, jämfört t.ex. med att jobba med att
administrera system dagligen. Även om jag anser mig vara något av en
''power-user'' med generellt goda kunskaper så har jag relativt lite erfarenhet
av nätverksadministration. Laborationen har gett ett bra tillfälle att öva på
det.
