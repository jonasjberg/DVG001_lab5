% ______________________________________________________________________________
%
% DVG001 -- Introduktion till Linux och små nätverk
%                              Inlämningsuppgift #5
% ~~~~~~~~~~~~~~~~~~~~~~~~~~~~~~~~~~~~~~~~~~~~~~~~~
% Author:   Jonas Sjöberg
%           tel12jsg@student.hig.se
%
% Date:     2016-04-27 -- 2016-05-09
%
% License:  Creative Commons Attribution 4.0 International (CC BY 4.0)
%           <http://creativecommons.org/licenses/by/4.0/legalcode>
%           See LICENSE.md for additional licensing information.
% ______________________________________________________________________________


\section{Del ett}
Den första delen innehåller kontroll av vilka tjänster och portar som är öppna
på datorn.


% ______________________________________________________________________________
\subsection{Uppgift 1}
\subsubsection{Uppgiftsbeskrivning}
Uppgiften är att kontrollera vilka portar som är öppna på vår server med
programmen \texttt{netstat} och \texttt{nmap}.  


\subsubsection{Lösning}
Först används programmet \texttt{netstat} för att visa information om
nätverksanslutningar. Körning visas i Programlistning~\ref{listing:netstat}.

% --verbose 
% --all 
% --tcp 
% --numeric

\shellsource{include/p1_netstat1}
					  {Körning av \texttt{netstat}.}
			 	    {listing:netstat}

Sedan körs programmet \texttt{nmap} med den lokala datorn som argument, utan
övriga flaggor. Körning av detta visas i Programlistning~\ref{listing:nmap}.

\begin{listing}[H]
  \shellcode{include/p1_nmap}
  \caption{Körning av \texttt{nmap}.}
  \label{listing:nmap}
\end{listing}

För att jämföra utmatningarnas innehåll kombineras innehållet i
Programlistning~\ref{listing:netstat} och Programlistning~\ref{listing:nmap}.
Kombinationen visas i Programlistning~\ref{listing:diff}

\begin{listing}[H]
  \shellcode{include/p1_nmap}
  \caption{Körning av \texttt{nnmap}.}
  \label{listing:nmap}
\end{listing}
