% ______________________________________________________________________________
%
% DVG001 -- Introduktion till Linux och små nätverk
%                              Inlämningsuppgift #5
% ~~~~~~~~~~~~~~~~~~~~~~~~~~~~~~~~~~~~~~~~~~~~~~~~~
% Author:   Jonas Sjöberg
%           tel12jsg@student.hig.se
%
% Date:     2016-04-27 -- 2016-05-09
%
% License:  Creative Commons Attribution 4.0 International (CC BY 4.0)
%           <http://creativecommons.org/licenses/by/4.0/legalcode>
%           See LICENSE.md for additional licensing information.
% ______________________________________________________________________________


\section{Del ett}
Den första delen innehåller undersökning av vilka tjänster och portar som är
öppna på datorn med hjälp av vanligt förekommande verktyg.


% ______________________________________________________________________________
\subsection{Uppgift 1}
\subsubsection{Uppgiftsbeskrivning}
Uppgiften är att kontrollera vilka portar som är öppna på vår server med
programmen \texttt{netstat} och \texttt{nmap}.


\subsubsection{Lösning}
Först används programmet \texttt{netstat} för att visa information om
nätverksanslutningar. Körning visas i Programlistning~\ref{listing:netstat}.

\shellsource{include/p1_netstat}
            {Körning av \texttt{netstat}.}
            {listing:netstat}


Sedan körs programmet \texttt{nmap} med den lokala datorn som argument, utan
övriga flaggor. Körning av detta visas i Programlistning~\ref{listing:nmap}.

\shellsource{include/p1_nmap}
            {Körning av \texttt{nmap}.}
            {listing:nmap}


För att jämföra utmatningarnas innehåll kombineras innehållet i
Programlistning~\ref{listing:netstat} och~\ref{listing:nmap}, vilket ger
resultatet som visas i Programlistning~\ref{listing:diff}.
Varje rad i listan en adress/port.  En del av informationen finns med i båda
programmens utmatning.  t.ex. vilket protokoll det rör sig om står under
kolumnen \texttt{Proto} i utmatningen från \texttt{netstat}, medan
\texttt{nmap} listar både port och protokoll (utan att specificera
\texttt{IPv4}/\texttt{IPv6}) under kolumnen \texttt{PORT}.
I utmatningen från \texttt{netstat} listas port tillsammans med
\texttt{IP}-adress} i kolumnen \texttt{Local Address}, samtidigt saknas
information om \texttt{IP}-adresser} helt i utmatningen från \texttt{nmap}.
Nämnvärt är också att \texttt{nmap} skriver ut information om ''services'',
alltså vilka program som lyssnar på en viss port, under kolumnen
\texttt{SERVICE}.

\shellsource{include/p1_diff}
            {Sammanslagning av informationen i
             Programlistning~\ref{listing:netstat} och~\ref{listing:nmap}.}
            {listing:diff}

