% ______________________________________________________________________________
%
% DVG001 -- Introduktion till Linux och små nätverk
%                              Inlämningsuppgift #5
% ~~~~~~~~~~~~~~~~~~~~~~~~~~~~~~~~~~~~~~~~~~~~~~~~~
% Author:   Jonas Sjöberg
%           tel12jsg@student.hig.se
%
% Date:     2016-04-27 -- 2016-05-09
%
% License:  Creative Commons Attribution 4.0 International (CC BY 4.0)
%           <http://creativecommons.org/licenses/by/4.0/legalcode>
%           See LICENSE.md for additional licensing information.
% ______________________________________________________________________________


\section{Del tre}
I den här delen upprättas grundläggande säkerhet med en brandvägg.


% ______________________________________________________________________________
\subsection{Uppgift 4}
\subsubsection{Uppgiftsbeskrivning}
Här används brandväggen \texttt{ufw} för att upprätta ett första grundläggande
skydd av servern från obehörigt tillträde.

\subsubsection{Lösning}
Till att börja med så installeras brandväggen \texttt{ufw} enligt 
Programlistning~\ref{listing:p3_ufw-install}.

\shellsource{include/p3_ufw-install}
            {Installation av brandväggen \texttt{ufw}.}
            {listing:p3_ufw-install}

Konfigurationsfilen \texttt{/etc/default/ufw} kontrolleras för
\texttt{IPv6}-stöd, som är aktiverat. Här finns också standardåtgärder för
matchningar mot regler och andra inställningar.

Regler för brandväggen matas in enligt labbinstruktionerna
\cite{dvg001:instruktionerLab5} och \cite{ubuntu:ufw}.

Den slutgiltiga konfigurationen av \texttt{ufw} visas tillsammans med en
körning av \texttt{nmap} i Programlistning~\ref{listing:p3_ufw-status_nmap}.

\shellsource{include/p3_ufw-status_nmap}
            {Slutgiltig konfiguration av \texttt{ufw} och körning av 
             \texttt{nmap} från servern.}
            {listing:p3_ufw-status_nmap}

Ytterligare en körning av \texttt{nmap} från maskinen ''\texttt{ProBookII}''
(\texttt{IP}-adress \texttt{192.168.1.107}) visas i
Programlistning~\ref{listing:p3_nmap}.

\shellsource{include/p3_nmap}
            {Körning av \texttt{nmap} mot servern från ''\texttt{ProBookII}''
             (\texttt{IP}-adress \texttt{192.168.1.107}).}
            {listing:p3_nmap}

Och slutligen visas samma sak från maskinen ''\texttt{ProBook-6465b}''
(\texttt{IP}-adress \texttt{192.168.1.110}) i
Programlistning~\ref{listing:p3_nmap2}.

\shellsource{include/p3_nmap2}
            {Körning av \texttt{nmap} mot servern från ''\texttt{ProBook-6465b}''
             (\texttt{IP}-adress \texttt{192.168.1.110}).}
            {listing:p3_nmap2}
