% ______________________________________________________________________________
%
% DVG001 -- Introduktion till Linux och små nätverk
%                              Inlämningsuppgift #5
% ~~~~~~~~~~~~~~~~~~~~~~~~~~~~~~~~~~~~~~~~~~~~~~~~~
% Author:   Jonas Sjöberg
%           tel12jsg@student.hig.se
%
% Date:     2016-04-27 -- 2016-05-09
%
% License:  Creative Commons Attribution 4.0 International (CC BY 4.0)
%           <http://creativecommons.org/licenses/by/4.0/legalcode>
%           See LICENSE.md for additional licensing information.
% ______________________________________________________________________________


\section{Del två}
I den här delen ska en webbserver installeras och konfigureras på servern.
Webbservern ska köras som användaren \texttt{www-data} och tillhöra gruppen
\texttt{www-data}.


% ______________________________________________________________________________
\subsection{Uppgift 2}
\subsubsection{Uppgiftsbeskrivning}
Här är uppgiften att installera \texttt{apache2} eller någon annan webbserver.
Innehållet i webbsidan ska sedan ändras (\texttt{index.html}) till något annat
än standardvärdet, varpå ändringen ska kontrolleras via lämplig \texttt{URL}.


\subsubsection{Lösning}
Här väljs för enkelhetens skull webbservern \texttt{apache}.
Valet av webbserver undersöks en del, massvis finns skrivet i frågan, t.ex.
\cite{webserver:compar1}, \cite{webserver:compar2}.

Först installeras webbservern \texttt{apache} från Debians paketarkiv med
\texttt{apt-get}.
Detta visas i Programlistning~\ref{listing:apache-install}.

\shellsource{include/p2_apache-install}
            {Installation av webbservern \texttt{apache}.}
            {listing:apache-install}


Efter installationen körs samma kontroll av öppna portar som i den första
uppgiften.  Först \texttt{netstat} i Programlistning~\ref{listing:2_netstat},
följt av \texttt{nmap} i Programlistning~\ref{listing:2_nmap}.

\shellsource{include/p2_netstat}
            {Körning av \texttt{netstat} efter installationen av \texttt{apache}.}
            {listing:2_netstat}

\shellsource{include/p2_nmap}
            {Körning av \texttt{nmap} efter installationen av \texttt{apache}.}
            {listing:2_nmap}


För att testa servern skrivs ett enkelt dokument i
\texttt{markdown}\cite{Gruber2013}-syntax som sedan kompileras till
\texttt{html} med hjälp av \texttt{pandoc}\cite{MacFarlane2013}.
Dokumentet visas visas i Programlistning~\ref{listing:p2_apache-testpage}.
En skärmdump på visning av det resulterande \texttt{html}-dokumentet visas i
Figur~\ref{fig:01}.

\begin{listing}[H]
\markdownfile{include/p2_apache-testpage.md}
\caption{Textfilen skriven med \texttt{markdown}-syntax som senare konverteras
         till \texttt{html} och visas renderad i Figur~\ref{fig:01}.}
\label{listing:p2_apache-testpage}
\end{listing}

\screenshot{include/p2_apache-screenshot}
           {Skärmdump på visning av den egna hemsidan.}
           {Skärmdump på den egna webbsidan (\texttt{index.html}) som
            tillhandahålls av \texttt{apache}.}
           {fig:01}


% ______________________________________________________________________________
\subsection{Uppgift 3}
\subsubsection{Uppgiftsbeskrivning}
Här ska en \texttt{NFS}-server och \texttt{NFS}-klient installeras och
konfigureras.


\subsubsection{Lösning}
Först installeras grundläggande paket enligt instruktioner.
Installationen visas i Programlistning~\ref{listing:p2_nfs-install}.
Programmen \texttt{nfs-common} och \texttt{rpcbind} är redan installerade
som standard med den version av Debian som körs i labmiljön.

\shellsource{include/p2_nfs-install}
            {Installation av \texttt{nfs-common}, \texttt{nfs-kernel-server}
             och \texttt{rpcbind.}}
            {listing:p2_nfs-install}


Status för \texttt{rpcbind} kontrolleras i
Programlistning~\ref{listing:p2_rpcbind-status}.

\shellsource{include/p2_rpcbind-status}
            {Statusinformation från tjänsten \texttt{rpcbind}.}
            {listing:p2_rpcbind-status}


Konfigurationsfilen \texttt{/etc/default/nfs-common} ändras så att servern kan
hantera översättning av användarnamn ordentligt.
Detta enligt labinstruktionerna\cite{dvg001:instruktionerLab5},
\cite{ubuntu:NFSv4howto} och \cite{ubuntu:settingupNFShowto}.

Den modifierade konfigurationsfilen visas i
Programlistning~\ref{listing:p2_nfs-common-config}.
Skillnaden är att \texttt{"yes"} lagts till på rad 16.

\shellsource{include/p2_nfs-common-config}
            {Den modifierade konfigurationsfilen \texttt{/etc/default/nfs-common}.}
            {listing:p2_nfs-common-config}


Efter att filen har modifierats startas tjänsten om för att förändringarna
ska träda i kraft, statusinformation skrivs sedan ut. Detta visas i
Programlistning~\ref{listing:p2_nfs-status}.

\shellsource{include/p2_nfs-status}
            {Tjänsten \texttt{nfs-kernel-server} startas om efter att
             konfigurationsfilen ändrats. Statusinformation skrivs sedan ut.}
            {listing:p2_nfs-status}


För att se vad som delas ut från servern används kommandot \texttt{showmount}
på serverns ''\texttt{localhost}'', \texttt{127.0.0.1}, detta visas i 
Programlistning~\ref{listing:p2_showmount}.

\shellsource{include/p2_showmount}
            {Undersökning av vilka filsystem som servern exporterar med 
             \texttt{NFS}.}
            {listing:p2_showmount}
