% ______________________________________________________________________________
%
% DVG001 -- Introduktion till Linux och små nätverk
%                              Inlämningsuppgift #5
% ~~~~~~~~~~~~~~~~~~~~~~~~~~~~~~~~~~~~~~~~~~~~~~~~~
% Author:   Jonas Sjöberg
%           tel12jsg@student.hig.se
%
% Date:     2016-04-27 -- 2016-05-09
%
% License:  Creative Commons Attribution 4.0 International (CC BY 4.0)
%           <http://creativecommons.org/licenses/by/4.0/legalcode>
%           See LICENSE.md for additional licensing information.
% ______________________________________________________________________________


\section{Del två}
I den här delen ska en webbserver installeras och konfigureras på servern.
Webbservern ska köras som användaren \texttt{www-data} och tillhöra gruppen
\texttt{www-data}.


% ______________________________________________________________________________
\subsection{Uppgift 2}
\subsubsection{Uppgiftsbeskrivning}
Här är uppgiften att installera \texttt{apache2} (eller någon annan befintlig
webbservrar) på vår server.
Innehållet i webbsidan ska sedan ändras (\texttt{index.html}) till något annat
än standardvärdet, varpå ändringen ska kontrolleras via lämplig \texttt{URL}.


\subsubsection{Lösning}
Här väljs för enkelhetens skull webbservern \texttt{apache}. 
Valet av webbserver undersöks en del, massvis finns skrivet i frågan, t.ex.
\cite{webserver:compar1}, \cite{webserver:compar2}.

Först installeras webbservern \texttt{apache} från Debians paketarkiv med
\texttt{apt-get}. 
Detta visas i Programlistning~\ref{listing:apache-install}.

\shellsource{include/p2_apache-install}
            {Installation av webbservern \texttt{apache}.}
            {listing:apache-install}


För att testa servern skrivs ett enkelt dokument i
\texttt{markdown}\cite{Gruber2013}-syntax som sedan kompileras till
\texttt{html} med hjälp av \texttt{pandoc}\cite{MacFarlane2013}.
Dokumentet visas visas i Programlistning~\ref{listing:p2_apache-testpage}.
En skärmdump på visning av det resulterande \texttt{html}-dokumentet visas i
Figur~\ref{fig:01}.

\begin{listing}[H]
\markdownfile{include/p2_apache-testpage.md}
\caption{Textfilen skriven i \texttt{markdown}-syntax som senare konverteras till \texttt{html}-sidan.}
\label{listing:p2_apache-testpage}
\end{listing}

\screenshot{include/p2_apache-screenshot}
           {Skärmdump på visning av den egna hemsidan.}
           {todo}
           {fig:01}




% ______________________________________________________________________________
\subsection{Uppgift 3}
\subsubsection{Uppgiftsbeskrivning}
Här ska en \texttt{NFS}-server och \texttt{NFS}-klient installeras och
konfigureras. 


\subsubsection{Lösning}
% TODO: ..
