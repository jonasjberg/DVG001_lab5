% ______________________________________________________________________________
%
% DVG001 -- Introduktion till Linux och små nätverk
%                              Inlämningsuppgift #5
% ~~~~~~~~~~~~~~~~~~~~~~~~~~~~~~~~~~~~~~~~~~~~~~~~~
% Author:   Jonas Sjöberg
%           tel12jsg@student.hig.se
%
% Date:     2016-04-27 -- 2016-05-09
%
% License:  Creative Commons Attribution 4.0 International (CC BY 4.0)
%           <http://creativecommons.org/licenses/by/4.0/legalcode>
%           See LICENSE.md for additional licensing information.
% ______________________________________________________________________________

\documentclass[11pt,a4paper]{article}

\usepackage[utf8]{inputenc}
\inputencoding{utf8}
\usepackage[swedish]{babel}
\usepackage[swedish]{isodate}
\usepackage[T1]{fontenc}

\usepackage{lmodern}
\usepackage{fullpage}

\usepackage{csquotes}               % Behövs av biblatex

\usepackage[natbib=true,
            style=ieee,
            backend=biber]{biblatex}
\addbibresource{tex/refs.bib}

\usepackage[binary-units=true]{siunitx}
\usepackage{float}
\usepackage{textcomp}
\usepackage{url}
\usepackage{graphicx}
%\usepackage{amssymb}
%\usepackage{amsmath}
\usepackage{amsfonts}
\usepackage{graphicx}
%\usepackage{microtype}

\usepackage[pdfusetitle,
            bookmarks=true,
            bookmarksnumbered=true,
            bookmarksopen=false,
            breaklinks=false,
            pdfborder={0 0 0},
            backref=false,
            colorlinks=false,
            hidelinks]{hyperref}

\newcommand{\screenshot}[4]{
\begin{figure}[H]
\centering
%\includegraphics[width=\linewidth]{#1}
\includegraphics[height=8.0cm]{#1}
\caption[#2]{#3}
\label{#4}
\end{figure}
}

\usepackage{minted}
\usemintedstyle{bw}

\usepackage{verbatim}
\usepackage{fancyvrb}
\usepackage{listings}

\newmintedfile[shellcode]{bash}{
%bgcolor=mintedbackground,
%fontfamily=tt,
breaklines=true,
fontsize=\footnotesize,
linenos=true,
numberblanklines=true,
numbersep=12pt,
numbersep=5pt,
%gobble= 0,
frame= lines,
%framerule= 0.4pt,
framesep=2mm,
funcnamehighlighting=true,
tabsize=4,
obeytabs=false,
mathescape=false
samepage=false,
showspaces=false,
showtabs=false,
texcl=false,
}

\newmintedfile[configfile]{linux-config}{
%bgcolor=mintedbackground,
%fontfamily=tt,
fontsize=\footnotesize,
linenos=true,
numberblanklines=true,
numbersep=12pt,
numbersep=5pt,
%gobble=0,
frame=lines,
%framerule=0.4pt,
framesep=2mm,
funcnamehighlighting=true,
tabsize=4,
obeytabs=false,
mathescape=false
samepage=false,
showspaces=false,
showtabs=false,
texcl=false,
}

\newmintedfile[markdownfile]{markdown}{
%bgcolor=mintedbackground,
%fontfamily=tt,
breaklines=true,
breakautoindent=true,
fontsize=\footnotesize,
linenos=true,
numberblanklines=true,
numbersep=12pt,
numbersep=5pt,
%gobble=0,
frame=lines,
%framerule=0.4pt,
framesep=2mm,
funcnamehighlighting=true,
tabsize=4,
obeytabs=false,
mathescape=false
samepage=false,
showspaces=false,
showtabs=false,
texcl=false,
}

\expandafter\def\csname PY@tok@err\endcsname{}
\expandafter\def\csname PYGdefault@tok@err\endcsname{\def\PYGdefault@bc##1{{\strut ##1}}}

\renewcommand\listingscaption{Programlistning}
\renewcommand\listoflistingscaption{Programlistningar}

\usepackage{booktabs}
\usepackage{longtable}

\usepackage{pdfpages}

\newcommand{\shellsource}[3]{
\begin{listing}[H]
\shellcode{#1}
\caption{#2}
\label{#3}
\end{listing}
}

\newcommand{\screenshot}[4]{
\begin{figure}[H]
\centering
\includegraphics[width=\linewidth]{#1}
%\includegraphics[height=8.0cm]{#1}
\caption[#2]{#3}
\label{#4}
\end{figure}
}


\title{\textsc{DVG001}                         \\
       Introduktion till Linux och små nätverk \\
       Laboration 5}

\author{                                 \\
  Jonas Sjöberg                          \\
  860224-xxxx                            \\
  Högskolan i Gävle                      \\
  \texttt{tel12jsg@student.hig.se}       \\
  \texttt{https://github.com/jonasjberg} \\
}

\date{}

\begin{document}
  \maketitle

  \begin{center}
    \begin{tabular}{l r}
      Utförd: & \isodate \printdate{2016-04-27} -- \printdate{2016-05-09} \\
      Kursansvarig lärare: & Anders Jackson                               \\
                           & Anders Hermansson
    \end{tabular}
  \end{center}

  \begin{abstract}
    Laboration i kursen \emph{DVG001 -- Introduktion till Linux och små
    nätverk} som läses på distans via Högskolan i Gävle under vårterminen 2016.
    Laborationen behandlar vidare nätverksadministration, vilket inkluderar
    kontroll av portar och tjänster, konfigurering av webbserver och filserver,
    samt installation och konfigurering av brandvägg.
  \end{abstract}

  \newpage
  %\hypersetup{linkcolor=black}
  \setcounter{tocdepth}{3}
  \tableofcontents

  \bigskip

  \listoffigures
  %\listoftables
  \listoflistings

  %\expandafter\def\csname PY@tok@err\endcsname{}

  \newpage
  % ______________________________________________________________________________
%
% DVG001 -- Introduktion till Linux och små nätverk
%                              Inlämningsuppgift #4
% ~~~~~~~~~~~~~~~~~~~~~~~~~~~~~~~~~~~~~~~~~~~~~~~~~
% Author:   Jonas Sjöberg
%           tel12jsg@student.hig.se
%
% Date:     2016-04-06 -- 2016-04-11
%
% License:  Creative Commons Attribution 4.0 International (CC BY 4.0)
%           <http://creativecommons.org/licenses/by/4.0/legalcode>
%           See LICENSE.md for additional licensing information.
% ______________________________________________________________________________


\section{Inledning}
% Skriv en kort inledning här som beskriver kortfattat vad rapporten handlar
% om. Den skall vara orienterande om Bakgrund och Syfte.


% ______________________________________________________________________________
\subsection{Bakgrund}
%    Beskriv lite mer ingående om bakgrunden till uppgiften, vad den handlar om.
Laborationen bygger vidare på de föregående laborationerna och behandlar vidare
grundläggande aspekter av kommunikation mellan datorer i ett nätverk.

Den virtuella maskin som skapades tidigare under kursens gång används under
laborationen.

% ______________________________________________________________________________
\subsection{Syfte}
% Skriv lite mer ingående om syftet med uppgiften.
Syftet med laborationen är att demonstrera och ge tillfälle till övning på
systemadministration, särskilt relaterat till nätverk.

% ______________________________________________________________________________
\subsection{Arbetsmetod}
% Hur kommer ni att arbeta?  Detta är en lite längre text än den rent
% orienterande texten i Planering och genomförande ovan.

Nedan följer en preliminär redogörelse för den experimentuppställning som används
under laborationen:

\begin{itemize}
  \item Laborationen utförs på en \texttt{ProBook-6545b} laptop som kör
        \texttt{Xubuntu 15.10} på kerneln \texttt{Linux 3.19.0-28}.

  \item Rapporten skrivs i \LaTeX\  som kompileras till pdf med \texttt{latexmk}.
        Detta sker på värdsystemet.

  \item Virtualisering sker med \texttt{Oracle VirtualBox} version
        \texttt{5.0.10\_Ubuntu r104061}.

  \item Utveckling av programkod och testkörning sker i gästsystemet som kör
        \texttt{Debian 7.3 (jessie)} på kerneln \texttt{Linux 3.16.0-4}.

  \item Både rapporten och koden skrivs med texteditorn \texttt{Vim}.

  \item För versionshantering av både rapporten och programkod används \texttt{Git}.
    \begin{itemize}
      \item Källkod till programmet och rapporten finns att hämta på:

            \url{https://github.com/jonasjberg/DVG001\_lab4}

      \item Hämta hem repon genom att exekvera följande från kommandoraden:
            
            \texttt{git clone git@github.com:jonasjberg/DVG001\_lab4.git}

    \end{itemize}
\end{itemize}



  % ______________________________________________________________________________
%
% DVG001 -- Introduktion till Linux och små nätverk
%                              Inlämningsuppgift #5
% ~~~~~~~~~~~~~~~~~~~~~~~~~~~~~~~~~~~~~~~~~~~~~~~~~
% Author:   Jonas Sjöberg
%           tel12jsg@student.hig.se
%
% Date:     2016-04-27 -- 2016-05-09
%
% License:  Creative Commons Attribution 4.0 International (CC BY 4.0)
%           <http://creativecommons.org/licenses/by/4.0/legalcode>
%           See LICENSE.md for additional licensing information.
% ______________________________________________________________________________


\section{Del ett}
Den första delen innehåller kontroll av vilka tjänster och portar som är öppna
på datorn.


% ______________________________________________________________________________
\subsection{Uppgift 1}
\subsubsection{Uppgiftsbeskrivning}
Uppgiften är att kontrollera vilka portar som är öppna på vår server med
programmen \texttt{netstat} och \texttt{nmap}.  


\subsubsection{Lösning}
Först används programmet \texttt{netstat} för att visa information om
nätverksanslutningar. 

--verbose 
--all 
--tcp 
--numeric

\begin{listing}[H]
  \shellcode{include/p1_netstat1}
  \caption{Körning av \texttt{netstat}.}
  \label{listing:netstat}
\end{listing}


Först används programmet \texttt{netstat} för att visa information om
nätverksanslutningar. 

  % ______________________________________________________________________________
%
% DVG001 -- Introduktion till Linux och små nätverk
%                              Inlämningsuppgift #5
% ~~~~~~~~~~~~~~~~~~~~~~~~~~~~~~~~~~~~~~~~~~~~~~~~~
% Author:   Jonas Sjöberg
%           tel12jsg@student.hig.se
%
% Date:     2016-04-27 -- 2016-05-09
%
% License:  Creative Commons Attribution 4.0 International (CC BY 4.0)
%           <http://creativecommons.org/licenses/by/4.0/legalcode>
%           See LICENSE.md for additional licensing information.
% ______________________________________________________________________________


\section{Del två}
I den här delen ska en webbserver installeras och konfigureras på servern.
Webbservern ska köras som användaren \texttt{www-data} och tillhöra gruppen
\texttt{www-data}.


% ______________________________________________________________________________
\subsection{Uppgift 2}
\subsubsection{Uppgiftsbeskrivning}
Här är uppgiften att installera \texttt{apache2} (eller någon annan befintlig
webbservrar) på vår server.
Innehållet i webbsidan ska sedan ändras (\texttt{index.html}) till något annat
än standardvärdet, varpå ändringen ska kontrolleras via lämplig \texttt{URL}.


\subsubsection{Lösning}
Här väljs för enkelhetens skull webbservern \texttt{apache}. 
Valet av webbserver undersöks en del, massvis finns skrivet i frågan, t.ex.
\cite{webserver:compar1}, \cite{webserver:compar2}.

Först installeras webbservern \texttt{apache} från Debians paketarkiv med
\texttt{apt-get}. 
Detta visas i Programlistning~\ref{listing:apache-install}.

\shellsource{include/p2_apache-install}
            {Installation av webbservern \texttt{apache}.}
            {listing:apache-install}


För att testa servern skrivs ett enkelt dokument i
\texttt{markdown}\cite{Gruber2013}-syntax som sedan kompileras till
\texttt{html} med hjälp av \texttt{pandoc}\cite{MacFarlane2013}.
Dokumentet visas visas i Programlistning~\ref{listing:p2_apache-testpage}.
En skärmdump på visning av det resulterande \texttt{html}-dokumentet visas i
Figur~\ref{fig:01}.

\begin{listing}[H]
\markdownfile{include/p2_apache-testpage.md}
\caption{Textfilen skriven i \texttt{markdown}-syntax som senare konverteras till \texttt{html}-sidan.}
\label{listing:p2_apache-testpage}
\end{listing}

\screenshot{include/p2_apache-screenshot}
           {Skärmdump på visning av den egna hemsidan.}
           {todo}
           {fig:01}




% ______________________________________________________________________________
\subsection{Uppgift 3}
\subsubsection{Uppgiftsbeskrivning}
Här ska en \texttt{NFS}-server och \texttt{NFS}-klient installeras och
konfigureras. 


\subsubsection{Lösning}
% TODO: ..

  % ______________________________________________________________________________
%
% DVG001 -- Introduktion till Linux och små nätverk
%                              Inlämningsuppgift #5
% ~~~~~~~~~~~~~~~~~~~~~~~~~~~~~~~~~~~~~~~~~~~~~~~~~
% Author:   Jonas Sjöberg
%           tel12jsg@student.hig.se
%
% Date:     2016-04-27 -- 2016-05-09
%
% License:  Creative Commons Attribution 4.0 International (CC BY 4.0)
%           <http://creativecommons.org/licenses/by/4.0/legalcode>
%           See LICENSE.md for additional licensing information.
% ______________________________________________________________________________


\section{Del tre}
I den här delen upprättas grundläggande säkerhet med en brandvägg.


% ______________________________________________________________________________
\subsection{Uppgift 4}
\subsubsection{Uppgiftsbeskrivning}
Här används brandväggen \texttt{ufw} för att upprätta ett första grundläggande
skydd av servern från obehörigt tillträde.

\subsubsection{Lösning}
Till att börja med så installeras brandväggen \texttt{ufw} enligt 
Programlistning~\ref{listing:p3_ufw-install}.

\shellsource{include/p3_ufw-install}
            {Installation av brandväggen \texttt{ufw}.}
            {listing:p3_ufw-install}

Konfigurationsfilen \texttt{/etc/default/ufw} kontrolleras för
\texttt{IPv6}-stöd, som är aktiverat. Här finns också standardåtgärder för
matchningar mot regler och andra inställningar.

Programlistning~\ref{listing:p3_ufw-config} visar hur regler för \texttt{SSH}
läggs till och aktivering av brandväggen, enligt
labbinstruktionerna\cite{dvg001:instruktionerLab5} och \cite{ubuntu:ufw}.

\shellsource{include/p3_ufw-config}
            {Regler för \texttt{SSH} läggs till, följt av att brandväggen
             aktiveras.}
            {listing:p3_ufw-config}

Den slutgiltiga konfigurationen av \texttt{ufw} visas tillsammans med en
körning av \texttt{nmap} i Programlistning~\ref{listing:p3_ufw-status_nmap}.

\shellsource{include/p3_ufw-status_nmap}
            {Slutgiltig konfiguration av \texttt{ufw} och körning av 
             \texttt{nmap} från servern.}
            {listing:p3_ufw-status_nmap}

Ytterligare en körning av \texttt{nmap} från maskinen ''\texttt{ProBookII}''
(\texttt{IP}-adress \texttt{192.168.1.107}) visas i
Programlistning~\ref{listing:p3_nmap}.

\shellsource{include/p3_nmap}
            {Körning av \texttt{nmap} mot servern från ''\texttt{ProBookII}''
             (\texttt{IP}-adress \texttt{192.168.1.107}).}
            {listing:p3_nmap}

Och slutligen visas samma sak från maskinen ''\texttt{ProBook-6465b}''
(\texttt{IP}-adress \texttt{192.168.1.110}) i
Programlistning~\ref{listing:p3_nmap2}.

\shellsource{include/p3_nmap2}
            {Körning av \texttt{nmap} mot servern från ''\texttt{ProBook-6465b}''
             (\texttt{IP}-adress \texttt{192.168.1.110}).}
            {listing:p3_nmap2}


  \newpage
  %% ______________________________________________________________________________
%
% DVG001 -- Introduktion till Linux och små nätverk
%                              Inlämningsuppgift #5
% ~~~~~~~~~~~~~~~~~~~~~~~~~~~~~~~~~~~~~~~~~~~~~~~~~
% Author:   Jonas Sjöberg
%           tel12jsg@student.hig.se
%
% Date:     2016-04-27 -- 2016-05-09
%
% License:  Creative Commons Attribution 4.0 International (CC BY 4.0)
%           <http://creativecommons.org/licenses/by/4.0/legalcode>
%           See LICENSE.md for additional licensing information.
% ______________________________________________________________________________


\section{Resultat}
Resultatet av laborationen är att vår laborationsmiljö fått ett grundläggande
skydd och dessutom möjlighet att agera både webb- och filserver.

% ~~~~~~~~~~~~~~~~~~~~~~~~~~~~~~~~~~~~~~~~~~~~~~~~~~~~~~~~~~~~~~~~~~~~~~~~~~~~~~
\section{Diskussion}
Som under tidigare laborationer kan många problem, detaljer och fallgropar
kräva vana och upprepad användning av verktyg och protokoll, för en verkligt
god kunskap om hur systemen fungerar. Mycket av det som krävs kan verka
godtyckligt och onödigt pillrigt, och det känns många gånger tydligt att
systemen byggts upp gradvis under väldigt många år. Samtidigt känner jag mig
mycket hemma i \texttt{UNIX}-miljön och skulle inte vilja byta ut det mot något
annat. Jag märker också att mitt synsätt blir helt annorlunda över tid, i 
och med att jag övar på att använda vissa program eller system. De ibland
svårtillgängliga, esoteriska, kryptiska kommandon och program en stöter på i
\texttt{UNIX}-miljön, blir med tid och övning fullständigt självklara och många
gånger kan den långa bakomliggande historien bidra till en ännu djupare
fascination för hela \texttt{UNIX}-kulturen.

Jag tycker att laborationen ger en bra introduktion och riktlinjer för fortsatt
fördjupning, som antagligen är nödvändig för att med tid verkligen greppa en
helhet.


% ~~~~~~~~~~~~~~~~~~~~~~~~~~~~~~~~~~~~~~~~~~~~~~~~~~~~~~~~~~~~~~~~~~~~~~~~~~~~~~
\section{Slutsatser}
Generellt tycker jag att resultaten varit goda, trots att jag själv privat
oftast t.ex. väljer att använda \texttt{gufw}, den grafiska frontenden till
\texttt{ufw} för enkelhetens skull, och därför har väldigt lite vana vid att
konfigurera just brandväggar på det vis som gjorts under laborationen. Dessutom
är det något jag oftast gör väldigt sällan, oftast en gång per installation av
ett operativsystem, så några tillfällen att verkligen bekanta sig med verktygen
och processen brukar inte presentera sig, till skillnad då mot någon som jobbar
med att administrera system dagligen. Även om jag anser mig vara något av en
''power-user'' med generellt goda kunskaper så har jag relativt lite erfarenhet
av nätverksadministration.


  \addcontentsline{toc}{section}{Referenser}
  \printbibliography{}

\end{document}
